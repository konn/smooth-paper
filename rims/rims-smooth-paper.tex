\documentclass[article]{jlreq}
\usepackage[theorem]{rims20j}
\usepackage{amsmath,amssymb}
\usepackage{graphicx}
\usepackage[backend=biber,bibstyle=numeric]{biblatex}
\addbibresource{references.bib}

\title{RIMS 講究録の原稿のタイトル}
\etitle{English title of the paper for RIMS K\={o}ky\={u}roku}
\author{\jaffil{DeepFlow 株式会社}
        \jname{石井大海}
        \jaddress{E-mail: {\ttfamily konn.jinro@gmail.com}}
        \ename{Hiromi ISHII}
        \eaffil{DeepFlow, Inc.}
  }
\date{}

\begin{document}

\maketitle

\begin{abstract}
  The connection between \emph{Automatic differentiation} and \emph{smooth infinitesimal analysis} has long been indicated, but an explicit explanation of their relationship has remained unfamiliar.
  In this paper, we give a brief explanation of this connection based on the concept of \emph{$C^\infty$-ring} and \emph{Weil algebra} introduced by Lawvere~\cite{lawvere1979categorical}.

  Based on this understanding, we give algorithms to realise automatic differentiation with \emph{higher infinitesimals}; more precisely, we will combine Automatic Differentiation and Gr\"{o}bner basis algorithm to achieve generalised automatic differentiation in arbitrary Weil algebra.
\end{abstract}

\section{準備}
さささしししすすすせせせそそそたたたちちちつつつてててととと
なななにににぬぬぬねねねのののはははひひひふふふへへへほほほ
まままみみみむむむめめめもももやややゆゆゆよよよわわわをををんんん。

\begin{Th}
  ここに定理を
\end{Th}

次の環境が用意されています。

\bigskip

\begin{tabular}{ll}
  Th & 定理 \\
  Pro & 命題 \\
  Lem & 補題 \\
  Cor & 系 \\
  Def & 定義 \\
  Hyp & 仮定 \\
  Nte & 記法 \\
  Exp & 例 \\
  Rem & 注意 \\
  Prob & 問題 \\
  Alg & アルゴリズム \\
  Proof & 証明 
\end{tabular}

\section{2番めのセクション}

\subsection{サブセクション一つ目}
あああ (\cite{abcd} を参照) ....

あああいいいうううえええおおおかかかきききくくくけけけこここ
さささしししすすすせせせそそそたたたちちちつつつてててととと
なななにににぬぬぬねねねのののはははひひひふふふへへへほほほ
まままみみみむむむめめめもももやややゆゆゆよよよわわわをををんんん。


\subsection{サブセクション二つ目}
いいい

%%%%%%%%%%%%%%%% Acknowledgements %%%%%%%%%%%%%%%%%%%
\begin{acknowledgements}
  この研究は〇〇の助成をうけています。
\end{acknowledgements}

%%%%%%%%%%%%%%%% Biblipgraphy %%%%%%%%%%%%%%%%%%%
\section*{\hfil{}参 考 文 献\hfill}
\printbibliography[heading=none]

\end{document}
