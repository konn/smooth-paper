\documentclass{article}
\usepackage[theorem]{rims20e}
\usepackage{amsmath,amssymb}
\usepackage{stmaryrd}
\usepackage{graphicx}
\usepackage{subfiles}
\usepackage{braket}
\usepackage{cleveref}
\usepackage[prologue,gray]{xcolor}
\usepackage{tikz}
\usetikzlibrary{matrix,decorations,arrows,calc,trees}
\usepackage{tcolorbox}
\tcbuselibrary{listings,xparse}
\usepackage{minted}
\setminted{style=borland}

\definecolor{ltblue}{rgb}{0,0.4,0.4}
\definecolor{dkblue}{rgb}{0,0.1,0.6}
\definecolor{dkgreen}{rgb}{0,0.35,0}
\definecolor{dkviolet}{rgb}{0.3,0,0.5}
\definecolor{dkred}{rgb}{0.5,0,0}
\definecolor{comment}{HTML}{444444}
\definecolor{keywd}{HTML}{8D00ED}
\definecolor{types}{HTML}{1F7B2F}
\definecolor{str}{HTML}{4070a0}
\definecolor{code-background}{gray}{0.8}
\definecolor{pragma}{HTML}{372A78}
\definecolor{num}{HTML}{40a070}
\definecolor{symb}{HTML}{000000}
\newcommand\symbmath[1]{{\ensuremath{\color{symb}#1}}}
\newcommand\typemath[1]{{\ensuremath{\ul{#1}}}}
\newcommand{\moncompose}{\mathbin{>\mkern-6mu=\mkern-6mu>}}
\newcommand\colmath[2]{{\ensuremath{\color{#1}#2}}}
\def\ul#1{{\underline{\bfseries #1}}}

\let\textt=\texttt
\newminted[code]{haskell}{%
mathescape,linenos,numbersep=5pt,frame=lines,framesep=2mm,%
}
\newminted[repl]{haskell}{%
mathescape,numbersep=5pt,frame=lines,framesep=2mm,%
}
\VerbatimFootnotes
\DeclareTotalTCBox{\hask}{v}{%
tcbox raise base,box align=base,verbatim,colback=lightgray,colframe=gray%
}{\mintinline[fontsize=\SMALL]{haskell}{#1}}
\let\haskinline=\hask

\usepackage[backend=biber,bibstyle=numeric]{biblatex}
\addbibresource{../references.bib}

\title{A Succinct Multivariate Lazy Multivariate Tower AD for Weil Algebra  Computation}
\author{Hiromi Ishii
        \affil{DeepFlow, Inc.}
  }
\endaddress{DeepFlow, Inc.
       \address{3-16-40 Tsuruse nishi, Fujimi-shi, Saitama prefecture, Japan.}
       \mail{h-ishii@math.tsukuba.ac.jp}
       \kanjidata{DeepFlow株式会社 \quad 石井大海}
     }
\date{}
\newcommand{\Rseries}{\R\llbracket\boldsymbol{X}\rrbracket}

\begin{document}

\maketitle

\begin{abstract}
  We propose a functional implementation of \emph{Multivariate Tower Automatic Differentiation}.
  Our implementation is intended to be used in implementing $C^\infty$-structure computation of an arbitrary Weil algebra, which we discussed in \cite{Ishii:2021vw}.
\end{abstract}

\section{Introduction}
\emph{Automatic Differentiation} (AD) is known as a powerful technique to compute differential coefficients of a given (piecewise) smooth function efficiently and accurately.
In the upcoming paper~\cite{Ishii:2021vw}, the author proposed to use \emph{$C^\infty$}-rings and Weil algebras to provide a modular and exprresive framework for forward-mode automatic difrerentiation.
There, compute the $C^\infty$-structure of an arbitrary Weil algebra as a quotient of that of the formal power series ring $\Rseries$.
The $C^\infty$-structure of $\Rseries$ was then computed via \emph{multivariate tower AD}.
It can be implemented in various ways, such as Lazy Multivariate Tower AD~\cite{Pearlmutter:2007aa}, or nested Sparse Tower AD~\cite[{module \texttt{Numeric.AD.Rank1.Sparse}}]{Kmett:2010aa}.

Theoretically, such existing methods can be used to compute the $C^\infty$-structure of $\Rseries$.
However, these methods are somewhat complex and not optimised for our purpose.
In this paper, we will propose another implementation of Lazy Multivariate Tower-Mode AD using tree representation and exploiting smoothness to save memory consumption.
Our method can be seen as aforementioned existing implementations~\cite{Pearlmutter:2007aa,Kmett:2010aa}.

% \begin{tabular}{ll}
%   Th & 定理 \\
%   Pro & 命題 \\
%   Lem & 補題 \\
%   Cor & 系 \\
%   Def & 定義 \\
%   Hyp & 仮定 \\
%   Nte & 記法 \\
%   Exp & 例 \\
%   Rem & 注意 \\
%   Prob & 問題 \\
%   Alg & アルゴリズム \\
%   Proof & 証明 
% \end{tabular}

\printbibliography

\end{document}
