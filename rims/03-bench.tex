%! TeX root = rims-smooth-paper.tex
\documentclass[rims-smooth-paper.tex]{subfiles}
\begin{document}
\section{Benchmarks}
\label{sec:bench}
We report several benchmarks on the proposed method with the existing implementation in the \texttt{ad}~\cite{Kmett:2010aa} package.

The code used in the benchmarks is implemented as a Haskell library and hosted on GitHub\footnote{\url{https://github.com/konn/smooth/tree/d3386a15c97f23d4071b09f97b7bc87f2c4b1da4}}.
All the benchmark code was compiled with GHC 8.10.4 with the flag \texttt{-threaded -O2}.
We ran the benchmark suites on a virtual Linux environment available on GitHub Actions (Standard\_DS2\_v2 Azure instance) with two Intel Xeon Platinum 8171M virtual CPUs (2.60GHz) and 7 GiB of RAM.
The Gauge framework~\cite{Hanquez:2019wk} was used to report the run-time speed.

\subsection{Tower Automatic Differentiation}
In this subsection, we compare the run-time speed and memory consumption of the existing multivariate tower AD and the proposed method.
In univariate case, we compare our proposed method with two existing implementations: \hask{Sparse} and \hask{diffs} both from \texttt{ad} package.
The former is the generic implementation of multivariate tower AD, and the former is speciailised implementation for the univariate case.

\begin{figure}[htbp]
\begin{center}
    \begin{tikzpicture}
    \begin{axis}[%
        legend columns=-1,
        legend entries={{\texttt{Sparse}},\textt{diffs},our method},
        legend to name=leg:time-tower,
        xlabel={Order},ylabel={Time (sec)},%
        title=Identity,%
      ]
      \addplot[mark=*] table[col sep=comma,x=size,y=AD] 
        {bench-results/multdiffupto-speed/identity.csv};
      \addplot[mark=o] table[col sep=comma,x=size,y=AD-diffs] 
        {bench-results/multdiffupto-speed/identity.csv};
      \addplot[mark=x] table[col sep=comma,x=size,y=STower] 
        {bench-results/multdiffupto-speed/identity.csv};
    \end{axis}
  \end{tikzpicture}
  %
  \begin{tikzpicture}
  \begin{axis}[%
      xlabel={Order},ylabel={Time (sec)},%
      title=$e^x$,%
    ]
    \addplot[mark=*] table[col sep=comma,x=size,y=AD] 
      {bench-results/multdiffupto-speed/exp-x.csv};
    \addplot[mark=o] table[col sep=comma,x=size,y=AD-diffs] 
      {bench-results/multdiffupto-speed/exp-x.csv};
    \addplot[mark=x] table[col sep=comma,x=size,y=STower] 
      {bench-results/multdiffupto-speed/exp-x.csv};
  \end{axis}
\end{tikzpicture}

\begin{tikzpicture}
  \begin{axis}[%
      xlabel style={align=center},
      xlabel={Total degrees:\\$(0, 1), (2, 0), (2, 1), (2, 2), (3, 2)$\\
        $(4, 2), (3, 4), (5, 3), (3, 6), (6, 4)$},ylabel={Time (sec)},%
      title=$\sin x e^{y^2}$,%
    ]
    \addplot[mark=*] table[col sep=comma,x=size,y=AD] 
      {bench-results/multdiffupto-speed/sin-x-exp-y2.csv};
    \addplot[mark=x] table[col sep=comma,x=size,y=STower] 
      {bench-results/multdiffupto-speed/sin-x-exp-y2.csv};
  \end{axis}
\end{tikzpicture}
%
\begin{tikzpicture}
  \begin{axis}[%
      xlabel style={align=center},
      xlabel={Total degrees:\\
      {$\scriptsize (0, 0, 1), (1, 0, 1), (0, 1, 2), (1, 2, 1), (0, 3, 2), (2, 2, 2)$}\\
      {$\scriptsize (3, 2, 2), (3, 4, 1), (5, 3, 1), (2, 3, 5), (5, 4, 2), (3, 4, 5)$}},%
      ylabel={Time (sec)},%
      title=$\sin x e^{y^2 + z}$,%
    ]
    \addplot[mark=*] table[col sep=comma,x=size,y=AD] 
      {bench-results/multdiffupto-speed/sin-x-exp-y2-z.csv};
    \addplot[mark=x] table[col sep=comma,x=size,y=STower] 
      {bench-results/multdiffupto-speed/sin-x-exp-y2-z.csv};
  \end{axis}
\end{tikzpicture}

\ref{leg:time-tower}
\end{center}

\caption{Speed benchmarks. \texttt{Sparse} and \texttt{diffs} are existing implementation in \texttt{ad} package.\label{fig:tower-time-bench}}
\end{figure}

\begin{figure}[htbp]
  \begin{center}
      \begin{tikzpicture}
      \begin{axis}[%
          legend columns=-1,
          legend entries={{\texttt{Sparse}},\textt{diffs},our method},
          legend to name=leg:heap-tower,%
          xlabel={Order},ylabel={Allocation (bytes)},%
          title=identity,%
        ]
        \addplot[mark=*] table[col sep=comma,x=size,y=AD] 
          {bench-results/multdiffupto-heap/identity.csv};
        \addplot[mark=o] table[col sep=comma,x=size,y=AD-diffs] 
          {bench-results/multdiffupto-heap/identity.csv};
        \addplot[mark=x] table[col sep=comma,x=size,y=STower] 
          {bench-results/multdiffupto-heap/identity.csv};
      \end{axis}
    \end{tikzpicture}
    %
    \begin{tikzpicture}
    \begin{axis}[%
        xlabel={Order},ylabel={Allocation (bytes)},%
        title=$e^x$,%
      ]
      \addplot[mark=*] table[col sep=comma,x=size,y=AD] 
        {bench-results/multdiffupto-heap/exp-x.csv};
      \addplot[mark=o] table[col sep=comma,x=size,y=AD-diffs] 
        {bench-results/multdiffupto-heap/exp-x.csv};
      \addplot[mark=x] table[col sep=comma,x=size,y=STower] 
        {bench-results/multdiffupto-heap/exp-x.csv};
    \end{axis}
  \end{tikzpicture}
  
  \begin{tikzpicture}
    \begin{axis}[%
        xlabel style={align=center},
        xlabel={Total degrees:\\$(0, 1), (2, 0), (2, 1), (2, 2), (3, 2)$\\
          $(4, 2), (3, 4), (5, 3), (3, 6), (6, 4)$},ylabel={Allocation (bytes)},%
        title=$\sin x e^{y^2}$,%
      ]
      \addplot[mark=*] table[col sep=comma,x=size,y=AD] 
        {bench-results/multdiffupto-heap/sin-x-exp-y2.csv};
      \addplot[mark=x] table[col sep=comma,x=size,y=STower] 
        {bench-results/multdiffupto-heap/sin-x-exp-y2.csv};
    \end{axis}
  \end{tikzpicture}
  %
  \begin{tikzpicture}
    \begin{axis}[%
        xlabel style={align=center},
        xlabel={Total degrees:\\
        {$\scriptsize (0, 0, 1), (1, 0, 1), (0, 1, 2), (1, 2, 1), (0, 3, 2), (2, 2, 2)$}\\
        {$\scriptsize (3, 2, 2), (3, 4, 1), (5, 3, 1), (2, 3, 5), (5, 4, 2), (3, 4, 5)$}},%
        ylabel={Allocation (bytes)},%
        title=$\sin x e^{y^2 + z}$,%
      ]
      \addplot[mark=*] table[col sep=comma,x=size,y=AD] 
        {bench-results/multdiffupto-heap/sin-x-exp-y2-z.csv};
      \addplot[mark=x] table[col sep=comma,x=size,y=STower] 
        {bench-results/multdiffupto-heap/sin-x-exp-y2-z.csv};
    \end{axis}
  \end{tikzpicture}

  \ref{leg:heap-tower}
\end{center}
  \caption{Heap benchmarks. \texttt{Sparse} and \texttt{diffs} are existing implementation in \texttt{ad} package.\label{fig:tower-heap-bench}}
\end{figure}

\Cref{fig:tower-time-bench,fig:tower-heap-bench} show the run-time speed and heap allocation of the existing implementations (\hask{Sparse} and \hask{diffs} only for univariate case), and the proposed method.
In univaricate case, the existing implementation for univariate tower AD provided, i.e.\ \hask{diffs} from \texttt{ad} package, peforms the best both in terms of run-time speed and memory allocation.
Notably, our implementation outperforms \hask{Sparse} both in time and space.
In particular, although \hask{Sparse} performs linearly both in time and space when applied to the identity, our method performs eventually constantly.
This is because our actual implementation employs some heuristics to detect a function whose higher derivatives are eventually zero.

Compared to \hask{Sparse}, our proposed method performs always better both in univariate and multivariate cases.
In particular, \hask{Sparse} presents a steep quadratic growth both in time and space, our proposed method peroforms almost linearly.

In summary, our method performs really well both in terms of time and space in the multivariate case, although we need more optimisation in univariate cases.

\subsection{Weil Algebra Computation}
In this subsection, we compare the performances of Tower AD applied to Weil algebra computation as described in \cite{Ishii:2021vw}.

\begin{figure}[btp]
  \begin{center}
    \begin{tikzpicture}
      \begin{axis}[%
          footnotesize,
          legend columns=-1,
          legend entries={{\texttt{Sparse}},our method},
          legend to name=leg:time-weil,
          ylabel={Time (sec)},%
          title=Identity,%
          symbolic x coords={sparse,x + d,dense},
          enlarge x limits=0.25,
          xtick=data,
          ymin=0,
          ybar,
        ]
        \addplot[black,fill=white] table[col sep=comma,x=size,y=AD] 
          {bench-results/liftWeil-speed/identity.csv};
        \addplot[black,fill=gray] table[col sep=comma,x=size,y=STower] 
          {bench-results/liftWeil-speed/identity.csv};
      \end{axis}
    \end{tikzpicture}%
    \quad
    \begin{tikzpicture}
      \begin{axis}[%
          footnotesize,
          ylabel={Time (sec)},%
          title={$e^x$},%
          symbolic x coords={sparse,x + d,dense},
          enlarge x limits=0.25,
          xtick=data,
          ymin=0,
          ybar,
        ]
        \addplot[black,fill=white] table[col sep=comma,x=size,y=AD] 
          {bench-results/liftWeil-speed/exp-x.csv};
        \addplot[black,fill=gray] table[col sep=comma,x=size,y=STower] 
          {bench-results/liftWeil-speed/exp-x.csv};
      \end{axis}
    \end{tikzpicture}
    \quad%
    \begin{tikzpicture}
      \begin{axis}[%
          footnotesize,
          ylabel={Time (sec)},%
          title={$\sin x e^{y^2 + z}$},%
          symbolic x coords={sparse,x + d,dense},
          enlarge x limits=0.25,
          xtick=data,
          ymin=0,
          ybar,
        ]
        \addplot[black,fill=white] table[col sep=comma,x=size,y=AD] 
          {bench-results/liftWeil-speed/sin-x-exp-y2-z.csv};
        \addplot[black,fill=gray] table[col sep=comma,x=size,y=STower] 
          {bench-results/liftWeil-speed/sin-x-exp-y2-z.csv};
      \end{axis}
    \end{tikzpicture}
  
  \ref{leg:time-weil}

  \texttt{Sparse} is the existing implementation in \texttt{ad} package.
  As an input, the ``sparse'' gives $1$ , ``x + d'' as it is, and ``dense'' $x + (\text{sum of all nonzero basis})$.
  
  \caption{Speed benchmarks for Weil algebra $W = \R[x,y]/(x^3 - y^2, y^3)$. \label{fig:weil-time-bench}}
  \end{center}
\end{figure}

\begin{figure}[btp]
  \begin{center}
    \begin{tikzpicture}
      \begin{axis}[%
          footnotesize,
          legend columns=-1,
          legend entries={{\texttt{Sparse}},our method},
          legend to name=leg:time-weil,
          ylabel={Alloc (bytes)},%
          title=Identity,%
          symbolic x coords={sparse,x + d,dense},
          enlarge x limits=0.25,
          xtick=data,
          ymin=0,
          ybar,
        ]
        \addplot[black,fill=white] table[col sep=comma,x=size,y=AD] 
          {bench-results/liftWeil-heap/identity.csv};
        \addplot[black,fill=gray] table[col sep=comma,x=size,y=STower] 
          {bench-results/liftWeil-heap/identity.csv};
      \end{axis}
    \end{tikzpicture}
    \quad
    \begin{tikzpicture}
      \begin{axis}[%
          footnotesize,
          ylabel={Alloc (bytes)},%
          title={$e^x$},%
          symbolic x coords={sparse,x + d,dense},
          enlarge x limits=0.25,
          xtick=data,
          ymin=0,
          ybar,
        ]
        \addplot[black,fill=white] table[col sep=comma,x=size,y=AD] 
          {bench-results/liftWeil-heap/exp-x.csv};
        \addplot[black,fill=gray] table[col sep=comma,x=size,y=STower] 
          {bench-results/liftWeil-heap/exp-x.csv};
      \end{axis}
    \end{tikzpicture}
    \quad%
    \begin{tikzpicture}
      \begin{axis}[%
          footnotesize,
          ylabel={Alloc (bytes)},%
          title={$\sin x e^{y^2 + z}$},%
          symbolic x coords={sparse,x + d,dense},
          enlarge x limits=0.25,
          xtick=data,
          ymin=0,
          ybar,
        ]
        \addplot[black,fill=white] table[col sep=comma,x=size,y=AD] 
          {bench-results/liftWeil-heap/sin-x-exp-y2-z.csv};
        \addplot[black,fill=gray] table[col sep=comma,x=size,y=STower] 
          {bench-results/liftWeil-heap/sin-x-exp-y2-z.csv};
      \end{axis}
    \end{tikzpicture}
  
  \ref{leg:time-weil}

  \texttt{Sparse} is the existing implementation in \texttt{ad} package.
  As an input, the ``sparse'' gives $1$ , ``x + d'' as it is, and ``dense'' $x + (\text{sum of all nonzero basis})$.
  
  \caption{Heap benchmarks for Weil algebra $W = \R[x,y]/(x^3 - y^2, y^3)$. \label{fig:weil-heap-bench}}
  \end{center}
\end{figure}

\Cref{fig:weil-time-bench,fig:weil-heap-bench} present the time and space performance of Weil algebra computation based on the existing implementation (\texttt{Sparse}) and the proposed method.
In particular, we evaluated three functions (the identity, $e^x$, and $\sin x \cdot e^{y^2 + z}$) on a Weil algebra $W = \R[x,y]/(x^3 - y^2, y^3)$.
We feed three types of inputs: a sparse input ($1$), $1 + d$, and dense input $1 + d_1 + d_2 + d_1^2 + d_1 d_2 + d_2^2$.
In any case, our proposed method largely outperforms the existing implementation.
\end{document}
