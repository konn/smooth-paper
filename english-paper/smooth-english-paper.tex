\documentclass[%
  sigconf,authorversion,screen]{acmart}
\usepackage{amsmath}
\usepackage{amsmath,amssymb}
\usepackage{stmaryrd}
\usepackage{graphicx}
\usepackage{subfiles}
\usepackage{braket}
\usepackage{cleveref}
\usepackage[prologue,gray]{xcolor}
\usepackage{tikz}
\usetikzlibrary{matrix,decorations,arrows,calc,trees}
\usepackage{tcolorbox}
\tcbuselibrary{listings,xparse}
\usepackage{minted}
\setminted{style=borland}

\definecolor{ltblue}{rgb}{0,0.4,0.4}
\definecolor{dkblue}{rgb}{0,0.1,0.6}
\definecolor{dkgreen}{rgb}{0,0.35,0}
\definecolor{dkviolet}{rgb}{0.3,0,0.5}
\definecolor{dkred}{rgb}{0.5,0,0}
\definecolor{comment}{HTML}{444444}
\definecolor{keywd}{HTML}{8D00ED}
\definecolor{types}{HTML}{1F7B2F}
\definecolor{str}{HTML}{4070a0}
\definecolor{code-background}{gray}{0.8}
\definecolor{pragma}{HTML}{372A78}
\definecolor{num}{HTML}{40a070}
\definecolor{symb}{HTML}{000000}
\newcommand\symbmath[1]{{\ensuremath{\color{symb}#1}}}
\newcommand\typemath[1]{{\ensuremath{\ul{#1}}}}
\newcommand{\moncompose}{\mathbin{>\mkern-6mu=\mkern-6mu>}}
\newcommand\colmath[2]{{\ensuremath{\color{#1}#2}}}
\def\ul#1{{\underline{\bfseries #1}}}

\let\textt=\texttt
\newminted[code]{haskell}{%
mathescape,linenos,numbersep=5pt,frame=lines,framesep=2mm,%
}
\newminted[repl]{haskell}{%
mathescape,numbersep=5pt,frame=lines,framesep=2mm,%
}
\VerbatimFootnotes
\DeclareTotalTCBox{\hask}{v}{%
tcbox raise base,box align=base,verbatim,colback=lightgray,colframe=gray%
}{\mintinline[fontsize=\SMALL]{haskell}{#1}}
\let\haskinline=\hask

\usepackage{pgfplots}
\usepackage[backend=biber,bibstyle=numeric]{biblatex}
\addbibresource{../references.bib}
\Crefformat{enumi}{(#2#1#3)}
\pgfplotsset{small,ylabel style={yshift=-12pt},%
  legend style={%
    outer sep=7pt,inner sep=5pt,%
    /tikz/every even column/.append style={column sep=10pt},%
  },
}

\title{A Succinct Multivariate Lazy Multivariate Tower AD for Weil Algebra Computation}
\newcommand{\Rseries}{\R\llbracket\boldsymbol{X}\rrbracket}
\date{}


\begin{document}

\title[Automatic Differentiation With Higher Infinitesimals]{Automatic Differentiation With Higher Infinitesimals, or Computational Smooth Infinitesimal Analysis in Weil Algebra}

\author{Hiromi ISHII}
\email{h-ishii@math.tsukuba.ac.jp}
\affiliation{%
  \institution{DeepFlow, Inc.}
  \streetaddress{3-16-40}
  \city{Fujimi-shi Tsuruse nishi}
  \state{Saitama prefecture}
  \country{Japan}
  \postcode{354-0026}
}

\renewcommand{\shortauthors}{Hiromi ISHII}

\begin{abstract}
  \frenchspacing
  We propose an algorithm to compute the $C^\infty$-ring structure of arbitrary Weil algebra, enabling us to do some analysis with \emph{higher inifinitesimals} numerically and symbolically.
  To that end, we first give a brief description of the (forward-mode) automatic differentiation in terms of $C^\infty$-rings.
  The notion of $C^\infty$-ring is introduced by Lawvere~\cite{lawvere1979categorical}, and used as the fundamental building block of \emph{smooth infinitesimal analysis} and \emph{synthetic differential geometry}~\cite{Moerdijk:1991aa}.
\end{abstract}

\begin{CCSXML}
<ccs2012>
    <concept>
        <concept_id>10010147.10010148.10010149.10010154</concept_id>
        <concept_desc>Computing methodologies~Hybrid symbolic-numeric methods</concept_desc>
        <concept_significance>500</concept_significance>
        </concept>
    <concept>
        <concept_id>10010147.10010148.10010149.10010150</concept_id>
        <concept_desc>Computing methodologies~Algebraic algorithms</concept_desc>
        <concept_significance>500</concept_significance>
        </concept>
    <concept>
        <concept_id>10010147.10010148.10010149.10010152</concept_id>
        <concept_desc>Computing methodologies~Symbolic calculus algorithms</concept_desc>
        <concept_significance>500</concept_significance>
        </concept>
    <concept>
        <concept_id>10002950.10003714.10003732.10003734</concept_id>
        <concept_desc>Mathematics of computing~Differential calculus</concept_desc>
        <concept_significance>500</concept_significance>
        </concept>
  </ccs2012>
\end{CCSXML}

\ccsdesc[500]{Computing methodologies~Hybrid symbolic-numeric methods}
\ccsdesc[500]{Computing methodologies~Algebraic algorithms}
\ccsdesc[500]{Computing methodologies~Symbolic calculus algorithms}
\ccsdesc[500]{Mathematics of computing~Differential calculus}
  
\keywords{automatic differentiation, %
  smooth infinitesimal analysis, %
  Weil algebras,%
  smooth algebras, $C^\infty$-rings, %
  symbolic-numeric algorihtms,
  symbolic differentiation, %
  Gr\"{o}bner basis, zero-dimensional ideals}

\maketitle

\section{Introduction}
\sloppy
\emph{Automatic Differentiation} is known as the method to calculate derivatives of (piecewise) smooth functions accurately and efficiently.
They are all quite different

\section{Preliminaries}

\section{Algorithms}

\begin{algorithm}[Weil algebra test]\label{alg:weil-test}
  \hspace{1em}\vspace{-.5em}
  \begin{description}
    \item[Input] An ideal $I \subseteq \mathbb{R}[X_1, \dots, X_n]$
    \item[Output] Returns the following data if $W = \mathbb{R}[\boldsymbol{X}]$ is a Weil algebra; otherwise \verb|No|.
    \begin{enumerate}
      \item Monomial basis $\set{\boldsymbol{w}_1, \dots, \boldsymbol{w}_\ell}$ of $W$,
      \item $M$, the multiplication table of $W$ in terms of the basis,
      \item $(k_1, \dots, k_n) \in \mathbb{N}^n$ such that $k_i$ is the maximum satisfying $X_i^{k_i} \notin I$ for each $i$, and
      \item $\NonVan_W$, a table of representations of non-vanishing monomials in $W$;
      i.e.\ for any $\alpha = (\alpha_1, \dots, \alpha_n) \in \N^n$, if $\alpha_i \leq k_i$ for all $i$, then $\NonVan_W(\X^\alpha) = (c_1, \dots, c_n) \in \R^k$ satisfies $[\X^\alpha]_I = \sum_i c_i \boldsymbol{w}_i$.
    \end{enumerate} 
  \end{description}

  \begin{alg}
G <- calcGroebnerBasis(I)
If @$I$@ is not zero-dimensional
  Return No
@$\set{\boldsymbol{w}_1, \dots, \boldsymbol{w}_\ell}$@ <- Monomial basis of @$I$@
@$M$@ <- the Multiplication table of @$W$@
For i in 1..n@\label{line:weil-test:radical-start}@
  @$p_i$@ <- the monic generator of @$I \cap \R[X_i]$@
  Return No If @$p_i$@ is not a monomial
  @$k_i$@ <- @$\deg(p_i) - 1$@@\label{line:weil-test:radical-end}@
@$\NonVan_W$@ <- {}
For @$\alpha$@ in @$\Set{\alpha \in \N^n | \alpha_i \leq k_i \; \forall i \leq \ell}$@
  @$c_1 \boldsymbol{w}_1 + \cdots + c_\ell \boldsymbol{w}_\ell$@ <- @$\rem{\X^\alpha}{G}$@
  @$\NonVan_W(\X^\alpha)$@ <- (@$c_1, \dots, c_\ell$@)
Return (@$\vec{\boldsymbol{w}}, M, \vec{k}, \NonVan_W$@)
\end{alg}
\end{algorithm}

\begin{theorem}
  \Cref{alg:weil-test} terminates and returns expected values.
\end{theorem}
\begin{proof}
  Algorithms to decide the zero-dimensionality and calculate their multiplication table is well-known (for detail we refer readers to Cox--Little--O'Shea~\cite[Chapter 2]{CLO:2005}).
  So the only non-trivial part is nilpotence detection (\Crefrange{line:weil-test:radical-start}{line:weil-test:radical-end}).
  But, again, this is just a variation of radical calculation algorithm for zero-dimensional ideals.
  Indeed, since each $\R[X_i]$ is a PID, we have $X_i^k \in I \cap R[X_i]$ iff $p_i \divs X_i^k$, hence $p_i$ must be a monomial iff $X_i$ is nilpotent in $W$.
\end{proof}

\section{Examples}

\section{Related and Future Works}

\section{Conclusion}

\begin{acks}
The author is grateful for Prof.\ Akira Terui, for encouraging  to write this paper.

This work was supported by the Research Institute for Mathematical Sciences,
an International Joint Usage/Research Center located in Kyoto University.
\end{acks}

\bibliographystyle{ACM-Reference-Format}
\bibliography{../references}

\end{document}
\endinput
