\documentclass[%
  sigconf,authorversion,screen]{acmart}
\usepackage{amsmath}
%%
%% \BibTeX command to typeset BibTeX logo in the docs
\AtBeginDocument{%
  \providecommand\BibTeX{{%
    \normalfont B\kern-0.5em{\scshape i\kern-0.25em b}\kern-0.8em\TeX}}}

% \setcopyright{acmcopyright}
\copyrightyear{2021}
\acmYear{2021}
% \acmDOI{10.1145/1122445.1122456}

% \acmConference[Woodstock '18]{Woodstock '18: ACM Symposium on Neural
%  Gaze Detection}{June 03--05, 2018}{Woodstock, NY}
% \acmBooktitle{Woodstock '18: ACM Symposium on Neural Gaze Detection,
%  June 03--05, 2018, Woodstock, NY}
% \acmPrice{15.00}
% \acmISBN{978-1-4503-XXXX-X/18/06}

%%\acmSubmissionID{123-A56-BU3}

\begin{document}

\title[Automatic Differentiation With Higher Infinitesimals]{Automatic Differentiation With Higher Infinitesimals, or Computational Smooth Infinitesimal Analysis in Arbitrary Weil Algebra}

\author{Hiromi ISHII}
\email{h-ishii@math.tsukuba.ac.jp}
\affiliation{%
  \institution{DeepFlow, Inc.}
  \streetaddress{3-16-40}
  \city{Fujimi-shi Tsuruse nishi}
  \state{Saitama prefecture}
  \country{Japan}
  \postcode{354-0026}
}

\renewcommand{\shortauthors}{Hiromi ISHII}

\begin{abstract}
  We propose a novel algorithm to compute the $C^\infty$-ring structure of arbitrary Weil algebra.

\end{abstract}

%%
%% The code below is generated by the tool at http://dl.acm.org/ccs.cfm.
%% Please copy and paste the code instead of the example below.
%%
\begin{CCSXML}
<ccs2012>
   <concept>
       <concept_id>10010147.10010148.10010149.10010154</concept_id>
       <concept_desc>Computing methodologies~Hybrid symbolic-numeric methods</concept_desc>
       <concept_significance>500</concept_significance>
       </concept>
   <concept>
       <concept_id>10010147.10010148.10010149.10010150</concept_id>
       <concept_desc>Computing methodologies~Algebraic algorithms</concept_desc>
       <concept_significance>500</concept_significance>
       </concept>
   <concept>
       <concept_id>10010147.10010148.10010149.10010152</concept_id>
       <concept_desc>Computing methodologies~Symbolic calculus algorithms</concept_desc>
       <concept_significance>500</concept_significance>
       </concept>
   <concept>
       <concept_id>10002950.10003714.10003732.10003734</concept_id>
       <concept_desc>Mathematics of computing~Differential calculus</concept_desc>
       <concept_significance>500</concept_significance>
       </concept>
 </ccs2012>
\end{CCSXML}

\ccsdesc[500]{Computing methodologies~Hybrid symbolic-numeric methods}
\ccsdesc[500]{Computing methodologies~Algebraic algorithms}
\ccsdesc[500]{Computing methodologies~Symbolic calculus algorithms}
\ccsdesc[500]{Mathematics of computing~Differential calculus}


\keywords{automatic differentiation, %
  smooth infinitesimal analysis, %
  Weil algebras,%
  smooth algebras, $C^\infty$-rings, %
  symbolic-numeric algorihtms,
  symbolic differentiation, %
  Gr\"{o}bner basis, zero-dimensional ideals}

\maketitle

\section{Introduction}

\section{Preliminaries}

\section{Algorithms}

\section{Benchmarks}

\section{Related and Future Works}

\section{Conclusion}

\begin{acks}
The author is grateful for Prof.\ Akira Terui, for encouraging  to write this paper.

This work was supported by the Research Institute for Mathematical Sciences,
an International Joint Usage/Research Center located in Kyoto University.
\end{acks}

\bibliographystyle{ACM-Reference-Format}
\bibliography{../references}

\end{document}
\endinput
